\begin{abstract}

Λόγω του μεγάλου όγκου πληροφοριών που κατακλύζει το διαδίκτυο, συχνά οι χρήστες
δυσκολεύονται να ξεχωρίσουν τις πληροφορίες που πραγματικά σχετίζονται με τα
ενδιαφέροντά τους. Επιπλέον, οι χρήστες έχουν πολύ διαφορετικά ενδιαφέροντα ή
προτιμήσεις που μπορούν να ληφθούν υπόψη ώστε να φιλτραριστούν ή να
ταξινομηθούν τα αποτελέσματα μιας ερώτησης με σκοπό το αποτέλεσμα να ικανοποιεί
τις εξατομικευμένες ανάγκες κάθε χρήστη. Η κατηγορία αυτών των συστημάτων
εξατομίκευσης ονομάζεται “Συστήματα Συστάσεων” {\en {(Recommender Systems)}}. Τα
Συστήματα Συστάσεων εκμεταλλεύονται τις ιδιαιτερότητες των χρηστών με σκοπό να τους 
διευκολύνουν στο να προσδιορίζουν ακριβέστερα τις πληροφορίες ή τις υπηρεσίες για
τις οποίες ενδιαφέρονται περισσότερο ή σχετίζονται με τις ανάγκες τους, κάνοντας
χρήση ειδικών αλγορίθμων. Οι αλγόριθμοι που χρησιμοποιούνται λαμβάνουν ως
είσοδο τα χαρακτηριστικά και τις προτιμήσεις των χρηστών ή τις σχέσεις μεταξύ των
χρηστών ή τα γνωρίσματα των προς σύσταση αντικειμένων και υπολογίζουν το
εκτιμώμενο ενδιαφέρον του χρήστη για κάθε αντικείμενο. Στην συνέχεια ταξινομούν ή
φιλτράρουν τα αντικείμενα με κριτήριο το εκτιμώμενο ενδιαφέρον. \\

Στο πλαίσιο της παρούσας διπλωματικής εργασίας σχεδιάστηκε και υλοποιήθηκε ένα 
σύστημα επεξεργασίας, ανάλυσης και ομαδοποίησης εγγράφων ειδήσεων του διαδικτύου, 
σχεδιασμένο ως εφαρμογή ειδησεογραφικού περιεχομένου που επιτρέπει στο χρήστη 
την περιήγηση μεταξύ των άρθρων μιας βάσης δεδομένων 
και λαμβάνοντας υπόψη τις επιλογές του, δηλαδή το προφίλ/ιστορικό του κάθε χρήστη, 
του προτείνει νέα άρθρα ειδήσεων που ταιριάζουν περισσότερο με τα ενδιαφέροντά του 
και δίνει τη δυνατότητα παρουσίασης των ειδήσεων οι οποίες σχετίζονται σημασιολογικά με αυτές που
έχει ήδη αναγνώσει.

\begin{keywords}
%\tl{RDF/S}, \tl{RQL}, \tl{Jxta}
\textit{Μηχανική Εκμάθηση, 
   %Συστήματα Συστάσεων, 
   Πολυεπίπεδο Σύστημα Συστάσεων Διαδικτυακών Εγγράφων Ειδήσεων, 
   Προσωποποιημένη Ανάκτηση Περιεχομένου, 
   Προσωποποιημένα Διαδικτυακά Έγγραφα Ειδήσεων, 
   Εξατομίκευση, Προφίλ Χρήστη, Μοντέλα Θεμάτων}
\end{keywords}
\end{abstract}


\chapter{\selectlanguage{greek}Συμπεράσματα και Μελλοντικές Επεκτάσεις}

\section{\selectlanguage{greek}Συμπεράσματα}

Καθώς όλο και περισσότερες πληροφορίες γίνονται διαθέσιμες στο διαδίκτυο, οι
χρήστες όλο και περισσότερο ψάχνουν απεγνωσμένα κάποια εργαλεία που θα τους
βοηθήσουν να φιλτράρουν αυτή τη ροή των πληροφοριών και να βρουν άρθρα
ειδήσεων που να τους ενδιαφέρουν. Ακριβώς σε ένα τέτοιο σενάριο, το σύστημα που
αναπτύχθηκε στο πλαίσιο αυτής της διπλωματικής εργασίας προσπαθεί να περιορίσει το πρόβλημα που
δημιουργείται από την διαρκή ροή ειδήσεων από διαφορετικές πηγές ενημέρωσης.
Αυτό που ουσιαστικά θέλουμε να δημιουργήσουμε είναι να παρέχουμε μια
εξατομικευμένη υπηρεσία συστάσεων για άρθρα ειδήσεων, στην οποία ο χρήστης 
θα επιλέγει τις κατηγορίες σχετικά με τις οποίες θέλει να ενημερώνεται και με βάση 
το προφίλ του και το ιστορικό χρήσης, το σύστημα θα παρέχει εξατομικευμένες συστάσεις 
που ταιριάζουν περισσότερο με τα ενδιαφέροντά του. \\

Στο πλαίσιο της εργασίας παρουσιάσαμε ένα σύστημα δημιουργίας εξατομικευμένων συστάσεων 
σε εφαρμογή διαδικτυακού περιεχομένου, 
η οποία λαμβάνει υπόψη το προφίλ και το ιστορικό των χρηστών του συστήματος για να 
προτείνει άρθρα ειδήσεων. 
Τα άρθρα αναπαρίστανται με τη βοήθεια θεματικών μοντέλων, 
δηλαδή μοντέλων για την ανακάλυψη θεμάτων που υπάρχουν σε μία συλλογή κειμένων. 
Τα προφίλ των χρηστών αναπαρίστανται μέσω μιας τριπλέτας αποτελούμενης από 
τα εξής χαρακτηριστικά: την κατανομή των θεμάτων των αναγνωσμένων άρθρων, 
τη λίστα χρηστών οι οποίοι έχουν παρόμοια πρότυπα πρόσβασης με τον εν λόγω χρήστη και 
τέλος, τη λίστα από ονοματισμένες οντότητες, δηλαδή λέξεις οι οποίες απαντούν σε φράσεις όπως 
``Τι συνέβη, ποιος εμπλέκεται, πότε συνέβη'' κλπ. 
Στόχος μας ήταν να βρούμε σύντομες περιγραφές των άρθρων της συλλογής και 
να εξερευνήσουμε τους συσχετισμούς μεταξύ των {\en {clusters}} (ή
των {\en {groups}}) άρθρων και του προφίλ του δοθέντος χρήστη, 
συγκρίνοντας την ομοιότητα των θεμάτων που “κρύβονται” μέσα στα άρθρα τους.
Οι σχέσεις μεταξύ αυτών των εννοιών εμπλουτίζουν τις παραπάνω αναπαραστάσεις 
και ενσωματώνονται στις διαδικασίες δημιουργίας συστάσεων. \\

Πιο συγκεκριμένα, συλλέξαμε άρθρα ειδήσεων τόσο από τον παγκόσμιο ιστό, όσο και από τη συλλογή {\en {Reuters}} του {\en {NLTK}} 
και τα αποθηκεύσαμε σε μία βάση δεδομένων. 
Το λειτουργικό μέρος του μηχανισμού εξάγει χρήσιμο κείμενο από αυτά, πραγματοποιεί μεθόδους εξαγωγής λέξεων
κλειδιών από κάθε άρθρο που υπάρχει στο σύστημα, εφαρμόζει τα θεματικά μοντέλα σε κάθε κατηγορία, γκρουπ και άρθρο, 
επιτρέποντας με αυτό τον τρόπο την περαιτέρω επεξεργασία τους βάσει σημασιολογικών συσχετίσεων μεταξύ των άρθρων.
Το ορατό στους χρήστες είναι ο δικτυακός τόπος που εμφανίζει τα άρθρα του συστήματος σημασιολογικά κατηγοριοποιημένα. 
Ο μηχανισμός προτάσεων του συστήματος, με βάση το προφίλ του χρήστη, 
παράγει ένα σύνολο εξατομικευμένων προτάσεων ειδήσεων που ταιριάζουν περισσότερο με τα
σημασιολογικά ενδιαφέροντα του χρήστη, κάνοντας δυνατή με αυτό τον τρόπο την
παρουσίαση ειδήσεων που σχετίζονται σημασιολογικά με αυτές που ήδη έχει
αναγνώσει. \\

Κατά την πειραματική αξιολόγηση του συστήματος, η ικανοποίηση κάθε χρήστη υπολογίσθηκε ως προς τα εξής τρία κριτήρια: 
συσχετισμός των προτεινόμενων άρθρων με τα πραγματικά του ενδιαφέροντα ({\en {Preference}}), 
ποικιλία της λίστας συστάσεων ({\en {Diversity}}) και 
κατάταξη των άρθρων της λίστας συστάσεων ({\en {Ordering}}). 
Επιπρόσθετα, παρουσιάστηκαν σενάρια χρήσης του συστήματος κατά τα οποία ο χρήστης 
είτε έχει αναγνώσει άρθρα από διάφορες κατηγορίες, είτε μόνο από μία κατηγορία 
και επιλέγει να δεχτεί τις συστάσεις του συστήματος. 
Διαπιστώθηκε ότι το σύστημα είναι σε θέση να καλύψει επαρκώς τις αναγνωστικές ανάγκες του χρήστη 
με βάση τα δεδομένα που του έχουν δοθεί ως είσοδος. 
Iδιαίτερα ικανοποιητικά είναι τα αποτελέσματα σχετικά με την ποικιλία της λίστας 
συστάσεων, όπου το σύστημα φαίνεται να εναρμονίζεται πλήρως με τα 
ενδιαφέροντα του χρήστη και να μην αφήνει καμία απ'τις ενδιαφέρουσες κατηγορίες 
άρθρων εκτός της τελικής σύστασης.\\

Από τα πειραματικά ευρήματα συμπεραίνουμε, επίσης, ότι το σύστημά μας έχει 
καλή απόδοση σε διαφορετικούς τύπους κειμένων και ιδιαίτερα, όταν τα
κείμενα είναι μικρά σε μέγεθος. Aυτό ερμηνεύεται μέσω του διανύσματος αναπαράστασης ενός κειμένου 
που προκύπτει από την εφαρμογή του αλγορίθμου {\en {LDA}}, 
στο οποίο επιλέγουμε τον αριθμό αντιπροσωπευτικών λέξεων απ'τις οποίες θα αποτελείται.
Έτσι, ένα τέτοιο διάνυσμα έχει μεγαλύτερες πιθανότητες να αποτυπώσει ορθότερα το νόημα ενός άρθρου, 
όταν το άρθρο αποτελείται από σχετικά μικρό αριθμό λέξεων. \\

Αξίζει να σημειώσουμε ότι τα θεματικά μοντέλα είναι ένα χρήσιμο εργαλείο εξερεύνησης. 
Τα θέματα παρέχουν μία περίληψη ενός σώματος κειμένων που είναι αδύνατο να γίνει με το χέρι. Η θεματική
ανάλυση μπορεί να ανακαλύψει συνδέσεις ανάμεσα και μέσα στα κείμενα που δεν είναι
φανερές με γυμνό μάτι και να βρει συσχετίσεις όρων που δε θεωρούνται δεδομένες. \\

% Ωστόσο, υπάρχουν περιθώρια βελτίωσης της απόδοσης του συστήματος,. \\
Πρόκληση για το σύστημά μας αποτελεί η αύξηση του πλήθους των άρθρων ανά κατηγορία στη βάση δεδομένων του συστήματος, 
καθώς και η αξιολόγηση των ομάδων που θα δημιουργηθούν από την εφαρμογή του αλγορίθμου {\en {k-means}} εντός κάθε κατηγορίας. \\

Τέλος, άξιο αναφοράς αποτελεί το πόσο βοηθητική υπήρξε η χρήση της 
γλώσσας προγραμματισμού {\en {Python}} σε μία τέτοια εφαρμογή 
και συγκεκριμένα η πλατφόρμα {\en {NLTK (Natural Language Toolkit)}}, 
μια πλατφόρμα με έτοιμα εργαλεία επεξεργασίας φυσικής γλώσσας. 

\section{\selectlanguage{greek}Μελλοντικές Επεκτάσεις}

Το σύστημα που αναπτύχθηκε στο πλαίσιο αυτής της διπλωματικής
εργασίας θα μπορούσε να βελτιωθεί και να επεκταθεί περαιτέρω,
τουλάχιστον ως προς τις κάτωθι κατευθύνσεις: \\

{\textbf {Συστάσεις σε ομάδες χρηστών}}\\
Η περιγραφή των προτιμήσεων με τη βοήθεια διανυσμάτων επιτρέπει το συνδυασμό
πολλαπλών προφίλ για τη δημιουργία ενός κοινού προφίλ για μια ομάδα χρηστών. \\

{\textbf {Ανατροφοδότηση του συστήματος με την αξιολόγηση του χρήστη}} \\
Το σύστημα μπορεί να δέχεται ως είσοδο την αξιολόγηση του χρήστη, 
αποθηκεύοντας στη βάση δεδομένων τα προτεινόμενα άρθρα που ο χρήστης βρήκε 
πραγματικά ενδιαφέροντα σε σχέση με ολόκληρη τη λίστα συστάσεων. 
Έτσι, μπορούν να πραγματοποιηθούν κάποιες πιο ισχυρές συνάψεις μεταξύ 
συγκεκριμένων άρθρων μέσα σε κάθε {\en {cluster}} και θεμάτων που 
διαφαίνονται από τις επιλογές ενός χρήστη, προκειμένου να λάβουμε βελτιωμένα αποτελέσματα 
σε επόμενη σύσταση. \\

{\textbf {Εφαρμογή συνάρτησης αξιολόγησης}} για την επιλογή ν-άδων σύστασης που 
προσφέρουν τη μέγιστη αύξηση ωφέλειας για το χρήστη (Εφαρμογή 
{\en {submodular}} μοντέλου συστάσεων με χρήση άπληστου προσεγγιστικού αλγορίθμου).

\begin{comment}
{\textbf {Πολυεπίπεδες υβριδικές συστάσεις}}
Το προφίλ ομάδων χρηστών χωρίζονται σε σημασιολογικούς τομείς. Καθένας από τους
τομείς αντιστοιχεί σε μια ομάδα και αντιπροσωπεύει ένα υποσύνολο των προτιμήσεων
του χρήστη που μοιράζεται με τους χρήστες που συμμετείχαν στη διαδικασία
δημιουργίας των ομάδων. Με την εισαγωγή επιπλέον δομής στα προφίλ των χρηστών
είναι δυνατό να οριστούν σχέσεις μεταξύ των χρηστών σε διαφορετικά επίπεδα,
δημιουργώντας με αυτό τον τρόπο πολύ-επίπεδες κοινότητες ενδιαφερόντων.
\end{comment}